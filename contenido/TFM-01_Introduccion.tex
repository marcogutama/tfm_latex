\chapter{Introducción}\label{chap:introduccion}

La transformación digital ha consolidado a las arquitecturas de microservicios y las metodologías DevOps como pilares fundamentales en el desarrollo de software moderno. Esta combinación permite a las organizaciones construir aplicaciones escalables y entregar valor de forma rápida y continua. Sin embargo, esta agilidad introduce nuevos y complejos desafíos de seguridad. La adopción de microservicios, si bien popular, incrementa la superficie de ataque de las aplicaciones \cite{Zafeiropoulos2023SecurityGaps}, y la automatización inherente a los pipelines de Integración y Despliegue Continuo (CI/CD), un pilar de DevOps, a menudo introduce riesgos si no se gestiona adecuadamente \cite{NIST2024CICDSecurity}.

Frente a esta problemática, la propuesta principal de esta investigación se enfoca en una solución innovadora: la integración de modelos de Inteligencia Artificial (IA) existentes directamente en los flujos de trabajo de CI/CD. Este enfoque busca mejorar radicalmente la forma en que se detectan y tratan las vulnerabilidades en el código Java de los microservicios. Es crucial aclarar que el objetivo no es la corrección automática de código, lo cual podría ser riesgoso, sino la generación automatizada de sugerencias de mitigación contextualizadas y de alta calidad. Estas sugerencias se presentarían al equipo de desarrollo para su revisión y validación, asegurando el control humano sobre cambios críticos.

Este capítulo introduce el problema, justifica la necesidad de una solución y plantea la pregunta de investigación que guía el trabajo. Finalmente, se presenta la estructura del documento, detallando los capítulos que componen la tesis para facilitar la comprensión del proceso de investigación, desarrollo y los resultados obtenidos.

\section{Justificación del trabajo}\label{sec:justificaciontrabajo}

La adopción de arquitecturas de microservicios se ha vuelto una tendencia principal en la industria del software, motivada por la búsqueda de mayor agilidad en el desarrollo, escalabilidad independiente de los componentes y mejor resiliencia del sistema. No obstante, esta descomposición en servicios añade una complejidad importante a la gestión de la seguridad. El problema fundamental que esta tesis busca resolver es la ineficiencia y la tendencia a errores de los métodos tradicionales para identificar y corregir vulnerabilidades en el amplio y dinámico entorno de microservicios. La naturaleza distribuida de estos sistemas incrementa la superficie de ataque, y la posible diversidad de tecnologías y equipos dificulta la aplicación de políticas de seguridad uniformes.

Las causas de este problema son diversas: la velocidad que exigen los ciclos de desarrollo ágil y DevOps a menudo deja poco margen para revisiones de seguridad manuales detalladas; la complejidad propia de las interacciones entre servicios puede ocultar vulnerabilidades sutiles; y la gran escala (cientos o miles de microservicios en organizaciones grandes) hace que la supervisión manual sea impracticable. La importancia de solucionar este problema es crítica, ya que las fallas de seguridad pueden tener consecuencias graves como pérdida de datos, daño a la reputación o incumplimiento de normativas. Ignorar estas alertas debido a la sobrecarga de información no es una opción, sino un riesgo latente que este trabajo busca mitigar.

La motivación principal de este trabajo nace de reconocer el potencial de la Inteligencia Artificial para enfrentar estos desafíos. La capacidad de los modelos de IA actuales para analizar grandes volúmenes de código, identificar patrones de vulnerabilidades y generar sugerencias de código coherentes ofrece una oportunidad única. Integrar esta capacidad en los pipelines CI/CD es un paso lógico hacia un enfoque de seguridad más inteligente y proactivo (DevSecOps). Por lo tanto, la contribución principal de esta tesis es el diseño, implementación y validación de un sistema modular que integra IA en pipelines CI/CD para generar sugerencias de mitigación de vulnerabilidades, buscando superar las limitaciones de las herramientas tradicionales en cuanto a contexto y accionabilidad.

Adicionalmente, se busca que la solución propuesta sea de código abierto, ofreciendo una alternativa a las herramientas comerciales existentes que pueden tener un alto costo \cite{ghas_docs, sonarqube_editions} o a las versiones comunitarias con funcionalidades limitadas. Un diferenciador clave del prototipo es su diseño modular, que permitirá al usuario seleccionar el modelo de IA subyacente, adaptando así la solución a diferentes contextos de presupuesto y privacidad.

\section{Planteamiento del problema}\label{sec:planteamiento_problema}

En el ciclo de vida actual del desarrollo de software, especialmente en organizaciones que utilizan DevOps y microservicios Java, se observa una importante brecha operativa. Existe una necesidad urgente de mecanismos que no solo detecten vulnerabilidades de forma temprana y continua (aplicando el principio "Shift Left"), sino que también ofrezcan a los equipos de desarrollo una guía clara, contextualizada y fiable para su corrección, integrada de forma fluida en sus flujos de trabajo automatizados.

Las soluciones actuales, aunque han progresado, presentan limitaciones. Las herramientas de Análisis Estático de Seguridad (SAST), si bien son esenciales, frecuentemente generan un gran volumen de alertas, incluyendo una cantidad significativa de falsos positivos. Este "ruido" dificulta la priorización, consume tiempo valioso de los desarrolladores y puede llevar a una "fatiga de alertas" que reduce la confianza en la herramienta \cite{Johnson2023UsabilitySAST}. La clasificación de estas alertas y la investigación de la mitigación adecuada sigue siendo un proceso manual intensivo, actuando como un cuello de botella que ralentiza el ciclo de retroalimentación.

Para responder a esta necesidad, se propone desarrollar un sistema que utiliza modelos de Inteligencia Artificial (IA) para el análisis integral de código, integrado como una etapa dentro del pipeline CI/CD. La finalidad de esta tesis es diseñar, construir y evaluar un prototipo funcional de este sistema. El prototipo enviará el código fuente a un modelo de IA seleccionado para un análisis exhaustivo que no solo identificará vulnerabilidades, sino que también evaluará la calidad general del código. Para validar su eficacia, los resultados se compararán con los de SonarQube, una herramienta de referencia industrial. Finalmente, los hallazgos y sugerencias de la IA se presentarán al desarrollador mediante reportes claros, generados como artefactos del pipeline.

Es fundamental reiterar que la propuesta se centra en la asistencia inteligente y no en la corrección automática. Se busca capacitar al desarrollador con información útil y de alta calidad, permitiéndole tomar decisiones informadas y mantener el control humano sobre el código.

\textit{Con base en lo anterior, la pregunta de investigación que guía este trabajo es: ¿En qué medida puede un sistema basado en modelos de IA, integrado en un pipeline CI/CD, complementar y enriquecer los resultados de una herramienta de análisis estático de seguridad (SAST) tradicional como SonarQube, mejorando la accionabilidad de las alertas y la calidad de las sugerencias de mitigación en microservicios Java?}

\section{Estructura de la memoria}\label{sec:estructura}

Para presentar de manera lógica y completa la investigación, los desarrollos y los resultados, este documento de tesis se organiza en los siguientes capítulos:

\begin{enumerate}
  \item \textbf{Introducción:} Presenta el tema central de la integración de IA en CI/CD para la seguridad de microservicios Java. Justifica la importancia del problema en el contexto de DevOps y microservicios, plantea la necesidad detectada y la solución propuesta, y describe la organización del resto del documento.

  \item \textbf{Contexto y estado del arte:} Proporciona el fundamento teórico y tecnológico. Profundiza en las arquitecturas de microservicios, sus implicaciones para la seguridad, y el papel de los pipelines CI/CD en DevOps. Revisa herramientas existentes para análisis de vulnerabilidades (SAST, DAST, SCA) y examina trabajos de investigación y soluciones relevantes que aplican IA a la seguridad del software, análisis de código u optimización de procesos DevOps. Busca identificar limitaciones actuales y destacar la originalidad de esta tesis.

  \item \textbf{Objetivos y metodología de trabajo:} Formaliza el alcance y plan de acción. Enuncia el objetivo general y lo desglosa en objetivos específicos medibles. Describe la metodología de investigación y desarrollo, especificando fases, tareas, criterios de selección de tecnologías y métodos de evaluación del prototipo.

  \item \textbf{Diseño e Implementación:} Es el núcleo técnico. Presenta la arquitectura detallada de la solución, explicando la integración de componentes. Justifica las decisiones de diseño y describe la implementación del prototipo, incluyendo la configuración de herramientas, la definición del Pipeline como Código (\texttt{Jenkinsfile}) y la lógica de interacción con la IA.

  \item \textbf{Evaluación y Resultados:} Presenta la evaluación empírica del prototipo. Describe experimentos, casos de prueba y métricas de rendimiento y efectividad. Analiza los resultados, discutiendo la calidad y utilidad de las sugerencias de la IA.

  \item \textbf{Conclusiones y Trabajo Futuro:} Sintetiza los hallazgos. Resume contribuciones, discute limitaciones y propone líneas de investigación futuras basadas en los resultados.
\end{enumerate}