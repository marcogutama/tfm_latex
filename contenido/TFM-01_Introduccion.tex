\chapter{Introducción}\label{chap:introduccion}
Este capítulo establece el marco teórico de la tesis, analizando los desafíos de seguridad en arquitecturas de microservicios y pipelines DevOps. Estudios recientes \cite{1}, \cite{2} destacan que la adopción de microservicios incrementa exponencialmente la superficie de ataque, mientras que la automatización DevOps prioriza la velocidad sobre controles de seguridad robustos \cite{3}.

La propuesta principal de esta investigación se enfoca en una solución innovadora: la integración de modelos de Inteligencia Artificial (IA) existentes en los flujos de trabajo automatizados de Integración Continua y Despliegue Continuo (CI/CD). Este enfoque busca mejorar la forma en que se detectan y tratan las vulnerabilidades en el código Java de los microservicios.

Se aclara desde el principio que el objetivo no es la corrección automática de código, lo cual podría ser riesgoso. En su lugar, se propone un paso intermedio que se alinea mejor con las prácticas seguras y la responsabilidad del desarrollador: la generación automatizada de sugerencias de mitigación contextualizadas y de alta calidad. Estas sugerencias se presentarían al equipo de desarrollo para su revisión, validación e implementación manual, asegurando así el control humano sobre cambios críticos en el código. Se describen los objetivos principales que guían el trabajo, entre los que destaca el diseño, la implementación y la validación de un prototipo funcional como resultado clave. Además, se anticipa la contribución de la tesis: no solo una mejora concreta en la seguridad de las aplicaciones y en la eficiencia operativa mediante la automatización de tareas de seguridad críticas (un pilar de DevSecOps), sino también una valiosa exploración de la interacción entre la IA y las prácticas DevOps. Esto posiciona el trabajo en un área relevante de la ingeniería de software moderna. Finalmente, se presenta la estructura del documento, detallando los capítulos para facilitar la comprensión del proceso de investigación, desarrollo y los resultados obtenidos.

\section{Justificación del trabajo}\label{sec:justificaciontrabajo}
La adopción de arquitecturas de microservicios se ha vuelto una tendencia principal en la industria del software, motivada por la búsqueda de mayor agilidad en el desarrollo, escalabilidad independiente de los componentes y mejor resiliencia del sistema. No obstante, esta descomposición en servicios añade una complejidad importante a la gestión de la seguridad. El problema fundamental que esta tesis busca resolver es la ineficiencia y la tendencia a errores de los métodos tradicionales (frecuentemente manuales o semi-manuales) para identificar y corregir vulnerabilidades en el amplio y dinámico entorno de microservicios de una aplicación actual. La naturaleza distribuida de estos sistemas incrementa la superficie de ataque, y la posible diversidad de tecnologías y equipos dificulta la aplicación de políticas de seguridad uniformes.

Las causas de este problema son diversas: la velocidad que exigen los ciclos de desarrollo ágil y DevOps a menudo deja poco margen para revisiones de seguridad manuales detalladas; la complejidad propia de las interacciones entre servicios puede ocultar vulnerabilidades sutiles; y la gran escala (cientos o miles de microservicios en organizaciones grandes) hace que la supervisión manual sea impracticable. La importancia de solucionar este problema es crítica. Las fallas de seguridad pueden tener consecuencias graves (pérdida de datos, daño a la reputación, incumplimiento de normativas), y la lentitud en la corrección de estas fallas aumenta el tiempo de exposición a riesgos. Por ello, mejorar la eficiencia y efectividad en la detección y mitigación de vulnerabilidades es clave para la sostenibilidad de las prácticas DevOps.

La motivación principal de este trabajo nace de reconocer el potencial de la Inteligencia Artificial para enfrentar estos desafíos complejos que dependen del conocimiento especializado. La capacidad de los modelos de IA actuales para analizar grandes volúmenes de código, identificar patrones (incluyendo los de vulnerabilidades) y generar texto coherente (como sugerencias de código) ofrece una oportunidad valiosa. Integrar esta capacidad directamente en los pipelines CI/CD, que son el núcleo de la automatización DevOps, es un paso lógico hacia un enfoque de seguridad más inteligente y proactivo (DevSecOps). El objetivo es ir más allá de la simple detección, ofreciendo una guía útil que permita a los desarrolladores construir software más seguro sin afectar la velocidad de desarrollo. Esta investigación se alinea con la tendencia industrial de AIOps y la aplicación de IA a la ingeniería de software, con la meta de aportar una contribución práctica y evaluada en el contexto específico de la seguridad de microservicios Java.

\section{Planteamiento del problema}\label{sec:planteamiento_problema}
En el ciclo de vida actual del desarrollo de software, especialmente en organizaciones que utilizan DevOps y arquitecturas de microservicios basadas en Java, se observa una importante brecha operativa. Existe una necesidad urgente de mecanismos que no solo detecten vulnerabilidades de seguridad de manera temprana y continua (aplicando el principio "Shift Left"), sino que también ofrezcan a los equipos de desarrollo una guía clara, contextualizada y fiable para su corrección. Todo esto debe estar integrado de forma fluida en sus flujos de trabajo automatizados (pipelines CI/CD).

Las soluciones actuales, aunque han progresado, tienen limitaciones en este aspecto. Las herramientas de Análisis Estático de Seguridad (SAST) y Análisis Dinámico de Seguridad (DAST), si bien son esenciales para la detección, frecuentemente generan un gran volumen de alertas (ruido). Muchas de estas alertas pueden ser falsos positivos o no tener suficiente información para una acción inmediata. La clasificación y priorización de estas alertas, así como la investigación de la mitigación adecuada para cada una, sigue siendo un proceso manual intensivo. Esto actúa como un cuello de botella, haciendo más lento el ciclo de retroalimentación y pudiendo retrasar la entrega de valor. La falta de sugerencias de corrección específicas para el código y el contexto del desarrollador reduce la agilidad y puede llevar a soluciones incompletas o incorrectas.

Para responder a esta necesidad, se propone desarrollar un enfoque nuevo: un sistema que coordine la colaboración entre herramientas de análisis de seguridad existentes y modelos de Inteligencia Artificial preentrenados, integrado como una etapa dentro del pipeline CI/CD. La finalidad de esta tesis es diseñar, construir y evaluar un prototipo funcional de este sistema. El sistema propuesto funcionará de la siguiente manera: primero, usará herramientas estándar (como SonarQube) para identificar posibles vulnerabilidades en el código fuente Java de los microservicios durante la fase de integración o construcción del pipeline. Segundo, las secciones de código señaladas como vulnerables se enviarán a un modelo de IA seleccionado (se explorarán opciones como Codex, Copilot, CodeBERT u otras alternativas, incluyendo opciones de código abierto si son viables). Tercero, el modelo de IA analizará el fragmento de código en su contexto y generará una o varias sugerencias de mitigación detalladas y específicas. Cuarto, estas sugerencias se presentarán al desarrollador mediante reportes claros, generados como artefactos del pipeline.

Es fundamental reiterar que la propuesta se centra en la asistencia inteligente y no en la corrección automática. Se busca capacitar al desarrollador con información útil y de alta calidad, permitiéndole tomar decisiones informadas y aplicar las correcciones de forma segura y eficiente. De esta manera, se mantiene la responsabilidad y el control humano sobre el código que llega a producción. El objetivo general es la automatización del proceso de generación de recomendaciones de seguridad contextualizadas dentro del ciclo de vida DevOps para microservicios Java, mejorando así la eficiencia y efectividad de las prácticas de DevSecOps.

\section{Estructura de la memoria}\label{sec:estructura}
Para presentar de manera lógica y completa la investigación, los desarrollos y los resultados, este documento de tesis se organiza en los siguientes capítulos:

\begin{enumerate}
\item \textbf{Introducción:} Presenta el tema central de la integración de IA en CI/CD para la seguridad de microservicios Java. Justifica la importancia del problema en el contexto de DevOps y microservicios, plantea la necesidad detectada y la solución propuesta, y describe la organización del resto del documento.

\item \textbf{Contexto y estado del arte:} Proporciona el fundamento teórico y tecnológico. Profundiza en las arquitecturas de microservicios, sus implicaciones para la seguridad, y el papel de los pipelines CI/CD en DevOps. Revisa herramientas existentes para análisis de vulnerabilidades (SAST, DAST, SCA) y examina trabajos de investigación y soluciones relevantes que aplican IA a la seguridad del software, análisis de código u optimización de procesos DevOps. Busca identificar limitaciones actuales y destacar la originalidad de esta tesis.

\item \textbf{Objetivos y metodología de trabajo:} Formaliza el alcance y plan de acción. Enuncia el objetivo general y lo desglosa en objetivos específicos medibles. Describe la metodología de investigación y desarrollo, especificando fases, tareas, criterios de selección de tecnologías y métodos de evaluación del prototipo.

\item \textbf{Diseño e Implementación:} Es el núcleo técnico. Presenta la arquitectura detallada de la solución, explicando la integración de componentes (repositorio, CI/CD, analizador, IA, reportes). Justifica decisiones de diseño y describe la implementación del prototipo, incluyendo configuración de herramientas y lógica de interacción con IA. Puede incluir diagramas y código relevante.

\item \textbf{Validación y Resultados:} Presenta la evaluación empírica del prototipo. Describe experimentos, casos de prueba (microservicios con vulnerabilidades) y métricas de rendimiento y efectividad (precisión de sugerencias, cobertura). Analiza los resultados, discutiendo la calidad y utilidad de las sugerencias de la IA.

\item \textbf{Conclusiones y Trabajo Futuro:} Sintetiza los hallazgos. Resume contribuciones, discute limitaciones y propone líneas de investigación futuras basadas en los resultados.
\end{enumerate}

Adicionalmente, el documento incluirá:

\begin{itemize}
\item \textbf{Referencias:}  Lista todas las fuentes bibliográficas citadas.
\item \textbf{Anexos:} Opcionalmente, incluye material complementario extenso (configuraciones, scripts, ejemplos de reportes).
\end{itemize}