\chapter{Introducción}\label{chap:introduccion}
Este capítulo presenta el contexto y la problemática central de la tesis, analizando los desafíos de seguridad en arquitecturas de microservicios y pipelines DevOps. La adopción de microservicios, si bien popular, incrementa la superficie de ataque de las aplicaciones \cite{Zafeiropoulos2023SecurityGaps}, y la automatización inherente a los pipelines de CI/CD, un pilar de DevOps, a menudo introduce riesgos de seguridad si no se gestiona adecuadamente \cite{NIST2024CICDSecurity}.

La propuesta principal de esta investigación se enfoca en una solución innovadora: la integración de modelos de Inteligencia Artificial (IA) existentes en los flujos de trabajo automatizados de Integración Continua y Despliegue Continuo (CI/CD). Este enfoque busca mejorar la forma en que se detectan y tratan las vulnerabilidades en el código Java de los microservicios.

Se aclara desde el principio que el objetivo no es la corrección automática de código, lo cual podría ser riesgoso. En su lugar, se propone un paso intermedio que se alinea mejor con las prácticas seguras y la responsabilidad del desarrollador: la generación automatizada de sugerencias de mitigación contextualizadas y de alta calidad. Estas sugerencias se presentarían al equipo de desarrollo para su revisión, validación e implementación manual, asegurando así el control humano sobre cambios críticos en el código. Se describen los objetivos principales que guían el trabajo, entre los que destaca el diseño, la implementación y la validación de un prototipo funcional como resultado clave. Además, se anticipa la contribución de la tesis: no solo una mejora concreta en la seguridad de las aplicaciones y en la eficiencia operativa mediante la automatización de tareas de seguridad críticas (un pilar de DevSecOps), sino también una valiosa exploración de la interacción entre la IA y las prácticas DevOps. Esto posiciona el trabajo en un área relevante de la ingeniería de software moderna. Finalmente, se presenta la estructura del documento, detallando los capítulos para facilitar la comprensión del proceso de investigación, desarrollo y los resultados obtenidos.

\section{Justificación del trabajo}\label{sec:justificaciontrabajo}
La adopción de arquitecturas de microservicios se ha vuelto una tendencia principal en la industria del software, motivada por la búsqueda de mayor agilidad en el desarrollo, escalabilidad independiente de los componentes y mejor resiliencia del sistema. No obstante, esta descomposición en servicios añade una complejidad importante a la gestión de la seguridad. El problema fundamental que esta tesis busca resolver es la ineficiencia y la tendencia a errores de los métodos tradicionales (frecuentemente manuales o semi-manuales) para identificar y corregir vulnerabilidades en el amplio y dinámico entorno de microservicios de una aplicación actual. La naturaleza distribuida de estos sistemas incrementa la superficie de ataque, y la posible diversidad de tecnologías y equipos dificulta la aplicación de políticas de seguridad uniformes.

Las causas de este problema son diversas: la velocidad que exigen los ciclos de desarrollo ágil y DevOps a menudo deja poco margen para revisiones de seguridad manuales detalladas; la complejidad propia de las interacciones entre servicios puede ocultar vulnerabilidades sutiles; y la gran escala (cientos o miles de microservicios en organizaciones grandes) hace que la supervisión manual sea impracticable. La importancia de solucionar este problema es crítica. Las fallas de seguridad pueden tener consecuencias graves (pérdida de datos, daño a la reputación, incumplimiento de normativas), y la lentitud en la corrección de estas fallas aumenta el tiempo de exposición a riesgos. Por ello, mejorar la eficiencia y efectividad en la detección y mitigación de vulnerabilidades es clave para la sostenibilidad de las prácticas DevOps.

La motivación principal de este trabajo nace de reconocer el potencial de la Inteligencia Artificial para enfrentar estos desafíos complejos que dependen del conocimiento especializado. La capacidad de los modelos de IA actuales para analizar grandes volúmenes de código, identificar patrones (incluyendo los de vulnerabilidades) y generar texto coherente (como sugerencias de código) ofrece una oportunidad valiosa. Integrar esta capacidad directamente en los pipelines CI/CD, que son el núcleo de la automatización DevOps, es un paso lógico hacia un enfoque de seguridad más inteligente y proactivo (DevSecOps). El objetivo es ir más allá de la simple detección, ofreciendo una guía útil que permita a los desarrolladores construir software más seguro sin afectar la velocidad de desarrollo. Esta investigación se alinea con la tendencia emergente de aplicar la Inteligencia Artificial a la ingeniería de software y, más concretamente, a las prácticas de DevSecOps, con la meta de aportar una contribución práctica y evaluada en el contexto específico de la seguridad de microservicios Java.

Adicionalmente, se busca que la solución propuesta sea de código abierto, ofreciendo una alternativa a las herramientas comerciales existentes que pueden tener un alto costo \cite{ghas_docs, sonarqube_editions} o a las versiones comunitarias con funcionalidades limitadas. Un diferenciador clave del prototipo es su diseño modular, que permitirá al usuario seleccionar el modelo de IA subyacente, ya sea uno de pago (por ejemplo, de OpenAI) o una alternativa de código abierto (como los disponibles en Hugging Face), adaptando así la solución a diferentes contextos de presupuesto y privacidad.

\section{Planteamiento del problema}\label{sec:planteamiento_problema}
En el ciclo de vida actual del desarrollo de software, especialmente en organizaciones que utilizan DevOps y arquitecturas de microservicios basadas en Java, se observa una importante brecha operativa. Existe una necesidad urgente de mecanismos que no solo detecten vulnerabilidades de seguridad de manera temprana y continua (aplicando el principio "Shift Left"), sino que también ofrezcan a los equipos de desarrollo una guía clara, contextualizada y fiable para su corrección. Todo esto debe estar integrado de forma fluida en sus flujos de trabajo automatizados (pipelines CI/CD).

Las soluciones actuales, aunque han progresado, tienen limitaciones en este aspecto. Las herramientas de Análisis Estático de Seguridad (SAST) y Análisis Dinámico de Seguridad (DAST), si bien son esenciales para la detección, frecuentemente generan un gran volumen de alertas (ruido), incluyendo una cantidad significativa de falsos positivos que dificultan la priorización y consumen tiempo de los desarrolladores \cite{11}. La clasificación y priorización de estas alertas, así como la investigación de la mitigación adecuada para cada una, sigue siendo un proceso manual intensivo. Esto actúa como un cuello de botella, haciendo más lento el ciclo de retroalimentación y pudiendo retrasar la entrega de valor. La falta de sugerencias de corrección específicas para el código y el contexto del desarrollador reduce la agilidad y puede llevar a soluciones incompletas o incorrectas.

Para responder a esta necesidad, se propone desarrollar un enfoque nuevo: un sistema que utiliza modelos de Inteligencia Artificial (IA) para el análisis integral de código, integrado como una etapa dentro del pipeline CI/CD. La finalidad de esta tesis es diseñar, construir y evaluar un prototipo funcional de este sistema. El sistema propuesto funcionará de la siguiente manera: primero, enviará el contenido completo de las clases Java a un modelo de IA seleccionado para un análisis exhaustivo. Los modelos de IA evaluados en este trabajo incluyen tanto modelos comerciales como Gemini, así como modelos de código abierto como DeepSeek y otros accesibles a través de la plataforma Ollama. Segundo, el modelo de IA no solo identificará vulnerabilidades, sino que también evaluará la calidad general del código, generando sugerencias de mejora y mitigación para los problemas detectados. Tercero, para validar la eficacia del análisis de la IA, se utilizará SonarQube como una herramienta de referencia comparativa. Se ejecutará un análisis de SonarQube sobre el mismo código, y sus resultados se compararán con los de la IA para evaluar la precisión en la detección de vulnerabilidades y la calidad de las sugerencias ofrecidas. Cuarto, los hallazgos y sugerencias de la IA se presentarán al desarrollador mediante reportes claros, generados como artefactos del pipeline.

Es fundamental reiterar que la propuesta se centra en la asistencia inteligente y no en la corrección automática. Se busca capacitar al desarrollador con información útil y de alta calidad, permitiéndole tomar decisiones informadas y aplicar las correcciones de forma segura y eficiente. De esta manera, se mantiene la responsabilidad y el control humano sobre el código que llega a producción. El objetivo general es la automatización del proceso de generación de recomendaciones de seguridad contextualizadas dentro del ciclo de vida DevOps para microservicios Java, mejorando así la eficiencia y efectividad de las prácticas de DevSecOps.

\textit{Con base en lo anterior, la pregunta de investigación que guía este trabajo es: ¿En qué medida puede un sistema basado en modelos de IA, integrado en un pipeline CI/CD, mejorar la detección de vulnerabilidades y la calidad de las sugerencias de mitigación en microservicios Java, en comparación con los resultados de una herramienta de análisis estático de seguridad (SAST) tradicional como SonarQube?}

\section{Estructura de la memoria}\label{sec:estructura}
Para presentar de manera lógica y completa la investigación, los desarrollos y los resultados, este documento de tesis se organiza en los siguientes capítulos:

\begin{enumerate}
\item \textbf{Introducción:} Presenta el tema central de la integración de IA en CI/CD para la seguridad de microservicios Java. Justifica la importancia del problema en el contexto de DevOps y microservicios, plantea la necesidad detectada y la solución propuesta, y describe la organización del resto del documento.

\item \textbf{Contexto y estado del arte:} Proporciona el fundamento teórico y tecnológico. Profundiza en las arquitecturas de microservicios, sus implicaciones para la seguridad, y el papel de los pipelines CI/CD en DevOps. Revisa herramientas existentes para análisis de vulnerabilidades (SAST, DAST, SCA) y examina trabajos de investigación y soluciones relevantes que aplican IA a la seguridad del software, análisis de código u optimización de procesos DevOps. Busca identificar limitaciones actuales y destacar la originalidad de esta tesis.

\item \textbf{Objetivos y metodología de trabajo:} Formaliza el alcance y plan de acción. Enuncia el objetivo general y lo desglosa en objetivos específicos medibles. Describe la metodología de investigación y desarrollo, especificando fases, tareas, criterios de selección de tecnologías y métodos de evaluación del prototipo.

\item \textbf{Diseño e Implementación:} Es el núcleo técnico. Presenta la arquitectura detallada de la solución, explicando la integración de componentes (repositorio, CI/CD, analizador, IA, reportes). Justifica las decisiones de diseño y describe la implementación del prototipo, incluyendo la configuración de herramientas y la lógica de interacción con la IA, ilustrando la solución con los diagramas y fragmentos de código relevantes.

\item \textbf{Evaluación y Resultados:} Presenta la evaluación empírica del prototipo. Describe experimentos, casos de prueba (microservicios con vulnerabilidades) y métricas de rendimiento y efectividad (precisión de sugerencias, cobertura). Analiza los resultados, discutiendo la calidad y utilidad de las sugerencias de la IA.

\item \textbf{Conclusiones y Trabajo Futuro:} Sintetiza los hallazgos. Resume contribuciones, discute limitaciones y propone líneas de investigación futuras basadas en los resultados.
\end{enumerate}

